%!TEX TS-program = xelatex
%!TEX options = -aux-directory=Debug -shell-escape -file-line-error -interaction=nonstopmode -halt-on-error -synctex=1 "%DOC%"
\documentclass{article}
\input{LaTeX-Submodule/template.tex}

% Additional packages & macros

% Header and footer
\newcommand{\unitName}{Electrical Engineering Mathematics}
\newcommand{\unitTime}{Semester 1, 2024}
\newcommand{\unitCoordinator}{Prof Scott McCue}
\newcommand{\documentAuthors}{Tarang Janawalkar}

\fancyhead[L]{\unitName}
\fancyhead[R]{\leftmark}
\fancyfoot[C]{\thepage}

% Copyright
\usepackage[
    type={CC},
    modifier={by-nc-sa},
    version={4.0},
    imagewidth={5em},
    hyphenation={raggedright}
]{doclicense}

\date{}

\begin{document}
%
\begin{titlepage}
    \vspace*{\fill}
    \begin{center}
        \LARGE{\textbf{\unitName}} \\[0.1in]
        \normalsize{\unitTime} \\[0.2in]
        \normalsize\textit{\unitCoordinator} \\[0.2in]
        \documentAuthors
    \end{center}
    \vspace*{\fill}
    \doclicenseThis
    \thispagestyle{empty}
\end{titlepage}
\newpage
%
\tableofcontents
\newpage
%
\section{Infinite Series}
\subsection{Sequences}
A sequence is an \textbf{ordered list} of numbers
\begin{equation*}
    a_1, \: a_2, \: a_3, \: \ldots, \: a_n, \: \ldots
\end{equation*}
denoted \({\left\{ a_n \right\}}_{n=1}^{\infty}\), where
\(n\) is the index of the sequence.
A sequence can be \textbf{finite} or \textbf{infinite}.
\subsection{Limits of Sequences}
An infinite sequence \(\left\{ a_n \right\}\) has a limit \(L\) if
\(a_n\) approaches \(L\) as \(n\) approaches infinity:
\begin{equation*}
    \lim_{n \to \infty} a_n = L
\end{equation*}
If such a limit exists, the sequence \textbf{converges} to \(L\).
Otherwise, the sequence \textbf{diverges}. Sequences that oscillate
between two or more values do not have a limit.
\subsection{Series}
Given a sequence \(\left\{ a_n \right\}\), we can construct a sequence
of \textbf{partial sums},
\begin{equation*}
    s_n = a_1 + a_2 + \cdots + a_n
\end{equation*}
denoted \(\left\{ s_n \right\}\), such that when \(\left\{ s_n \right\}\)
converges to a finite limit \(L\), that is,
\begin{equation*}
    \lim_{n \to \infty} s_n = L
\end{equation*}
the \textbf{infinite series} \(\sum_{n=1}^{\infty} a_n\) converges to \(L\).
Otherwise, the series \(\sum_{n=1}^{\infty} a_n\) diverges.
\subsubsection{Common Series}
Below are a list of common series that converge to a finite limit:
\begin{itemize}
    \item \textbf{Geometric Series}: A sum of the geometric progression
    \begin{equation*}
        \sum_{n=0}^{\infty} a r^n
    \end{equation*}
    converges when \(\abs*{r} < 1\), and diverges otherwise. When
    \(\abs*{r} < 1\),
    \begin{equation*}
        \sum_{n=0}^{\infty} a r^n = \frac{a}{1 - r}
    \end{equation*}
    \item \textbf{Harmonic Series}: A sum of the reciprocals of natural numbers
    \begin{equation*}
        \sum_{n=1}^{\infty} \frac{1}{n}
    \end{equation*}
    always diverges.
    \item \textbf{\(p\)-Series}: A sum of the reciprocals of \(p\)-powers of
    natural numbers
    \begin{equation*}
        \sum_{n=1}^{\infty} \frac{1}{n^p}
    \end{equation*}
    converges when \(p > 1\), and diverges otherwise. This series is
    closely related to the \textbf{Riemann Zeta Function}, and has
    exact values for even integers \(p\).
\end{itemize}
\subsection{Convergence Tests}
There are several tests to determine the convergence of an infinite series.
Note that these tests do not determine the value of the limit.
\subsubsection{Ratio Test}
Given the infinite series \(\sum_{n=1}^{\infty} a_n\), with
\begin{equation*}
    \rho = \lim_{n \to \infty} \abs*{\frac{a_{n+1}}{a_n}}
\end{equation*}
\begin{enumerate}[label=(\arabic*)]
    \item If \(\rho < 1\), the series converges.
    \item If \(\rho > 1\), the series diverges.
    \item If \(\rho = 1\), the test is inconclusive.
\end{enumerate}
\subsubsection{Alternating Series Test}
Given the infinite series \(\sum_{n=1}^{\infty} {\left( -1 \right)}^{n-1} b_n\),
the alternating series converges if the following conditions are met:
\begin{enumerate}[label=(\arabic*)]
    \item \(b_n > 0\) for all \(n\).
    \item \(b_{n+1} \leq b_n\) for all \(n\).
    \item \(\lim_{n \to \infty} b_n = 0\).
\end{enumerate}
\end{document}
